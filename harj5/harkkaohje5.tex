\documentclass[12pt,portrait,a4]{article}
\usepackage[finnish]{babel}
\usepackage[utf8]{inputenc}
\usepackage{multicol}
\usepackage{calc}
\usepackage{ifthen}
\usepackage{geometry}
\usepackage{amsmath,amsthm,amsfonts,amssymb}
\usepackage{color,graphicx,overpic}
\usepackage{hyperref}
\usepackage{enumitem}


\pdfinfo{
  /Title (harkkaohje5.pdf)
  /Creator (TeX)
  /Producer (pdfTeX 1.40.0)
  /Author (Risto Järvinen ja CEC ry)
  /Subject (Harkkaohje5)
  /Keywords (pdflatex, latex,pdftex,tex)}

% This sets page margins to .5 inch if using letter paper, and to 1cm
% if using A4 paper. (This probably isn't strictly necessary.)
% If using another size paper, use default 1cm margins.
\ifthenelse{\lengthtest { \paperwidth = 11in}}
    { \geometry{top=.5in,left=.5in,right=.5in,bottom=.5in} }
    {\ifthenelse{ \lengthtest{ \paperwidth = 297mm}}
        {\geometry{top=1cm,left=1cm,right=1cm,bottom=1cm} }
        {\geometry{top=1cm,left=1cm,right=1cm,bottom=1cm} }
    }

% Turn off header and footer
\pagestyle{empty}

% Redefine section commands to use less space
\makeatletter
\renewcommand{\section}{\@startsection{section}{1}{0mm}%
                                {-1ex plus -.5ex minus -.2ex}%
                                {0.5ex plus .2ex}%x
                                {\normalfont\large\bfseries}}
\renewcommand{\subsection}{\@startsection{subsection}{2}{0mm}%
                                {-1explus -.5ex minus -.2ex}%
                                {0.5ex plus .2ex}%
                                {\normalfont\normalsize\bfseries}}
\renewcommand{\subsubsection}{\@startsection{subsubsection}{3}{0mm}%
                                {-1ex plus -.5ex minus -.2ex}%
                                {1ex plus .2ex}%
                                {\normalfont\small\bfseries}}
\makeatother

% Don't print section numbers
\setcounter{secnumdepth}{0}


\setlength{\parindent}{0pt}
\setlength{\parskip}{0pt plus 0.5ex}


% -----------------------------------------------------------------------

\begin{document}
\raggedbottom

\begin{multicols}{2}
\setlength{\premulticols}{1pt}
\setlength{\postmulticols}{1pt}
\setlength{\multicolsep}{1pt}
\setlength{\columnsep}{2pt}

Nimi:\hrulefill

Op.num:\hrulefill

\end{multicols}


\section{Ohjelmointityökalut}

Ohjelmistokehitys on avoimen lähdekoodin aatteen lähde, joten ei tule
suurena yllätyksenä että Linux"-jakeluissa on mukana kattavat
kehitystyökalut.

\begin{itemize}
\item sudo apt-get install build-essential git gitk devscripts
\end{itemize}

Ikävä kyllä tämä harkkaohje kirjoitettiin koneella jossa oli jo satoja
ohjelmointikirjastoja asennettuna, joten tästä puuttuu varmaan tulevien
esimerkkien tarvitsemia kirjastoja.  Näiden setviminen on osa harjoitusta.
Tämän harjoituksen tekeminen ei tarvi ohjelmointitaitoja, mutta hyvinkin
paljon päättelyä ja uuden omaksumista.

\section{C ja Git, oma esimerkkiprojekti}

\begin{enumerate}
\item Konfiguroidaan ensin Gitille nimesi ja email"-osoitteesi.  Nämä tekevät
lähinnä logien lukemisesta helpompaa.
\begin{itemize}
\item \emph{git config --global user.name "Oma Nimi"}
\item \emph{git config --global user.email oma.nimi@aalto.fi}
\end{itemize}
Git tallentaa nämä kotihakemistoosi ~/.gitconfig tiedostoon.  Voit syöttää
tiedot sinne myös tekstieditorilla.

\item Tehdään oma projekti!  Luo hakemisto nimeltä ''oma'' ja siirry sinne:
\emph{mkdir oma ; cd oma}
\item Luodaan uusi git"-repositorio: \emph{git init}
\item Luodaan uusi lähdekooditiedosto: \emph{nano helloworld.c}
Kirjoita seuraavat sisään luomaasi tiedostoon:
\begin{verbatim}
/* Hello World program */
#include <stdio.h>
int main(void) {
        printf("Hello World\n");
        return 0;
}
\end{verbatim}
Varo kirjoitusvirheitä.  Ohjelmointikielet ovat yleensäkin täysin armottomia
syntaksivirheiden suhteen.  C"-ohjelmointi ei kuulu tämän harjoituksen tai
kurssin piiriin, vaan sitä saatte \emph{ELEC-A1100 C"-ohjelmoinnin
peruskurssilla}.

\item Käännä kokeeksi ohjelma komennolla: \emph{gcc -o helloworld -Wall helloworld.c}
(-o määrittelee ulostulotiedoston ja -Wall pyytää kääntäjältä raportoimaan
kaikki varoitukset)
\item Suorita kääntämäsi ohjelma: \emph{./helloworld}
(''./'' nimen alussa tarkoittaa että tarkoitat juuri tässä kansiossa olevaa
ohjelmaa.  Muutoin saisit varoituksen siitä että komentoa ei löydy polulta.)

\item Kommitoidaan tekemäsi kooditiedosto projektiin: \emph{git add
helloworld.c}
\emph{git commit -m ''Uusi kooditiedosto''}
(-m ''kommenti'' on tapa lisätä lyhyt kommentti kommitointiin.  Jos sen jättää
pois, git käynnistää tekstieditorin jotta voit kirjoittaa logientryn.)

\item Näet tekemäsi muutoksen logista komennolla: \emph{git log}

\item Hienon ohjelmasi tuloste saa kuitenkin välimerkkifetisistit
raivostumaan.  Korjaa ''Hello World'' viesti heidän vaatimaan muotoon
''Hello, World!'' tekstieditorilla.
\item Näet nyt tekemäsi muutoksen komennoilla \emph{git status} ja \emph{git diff}.
\item Kommitoi muutoksesi: \emph{git commit -m ''Välimerkit'' helloworld.c}

\item Lisätään projektiin Makefile, kääntämisen helpottamiseksi.  Luo
tekstieditorilla tiedosto ''Makefile'' ja lisää sinne seuraava sisältö:
\begin{verbatim}
helloworld: helloworld.c
        gcc -o helloworld -Wall helloworld.c
\end{verbatim}
\item Käännä nyt muokkaamasi versio komennolla: \emph{make}

\item Kommitoidaan Makefile: \emph{git add Makefile}\\
\emph{git commit -m ''Lisätään Makefile''}

\item Oikeastaan kun asiaa vähän harkitaan, välimerkkifetistien vaatimukset
eivät saa rajoittaa akateemista vapauttamme.  Perutaan heidän vaatima
muutos!  Käytetään gitin versionhallintavoimaa tämän tekemiseksi.
Katso \emph{git log} komennolla ''Välimerkit'' kommitin tunniste (se
heksanumerosotku sanan commit jälkeen) ja suorita komento: \emph{git revert
heksanumerosotku}
Git avaa tekstieditorin jolla voit tehdä logiviestin, mutta oletus on
riittävän hyvä, joten sulje editori.
\item Käännä nyt ohjelma \emph{make} komennolla ja verifioi että se toimii
\emph{./helloworld}.

\item Tarkastele miltä gitin tietokanta näyttää komennolla: \emph{gitk}
\end{enumerate}

Makefile on joustava mekanismi kääntää isojakin projekteja.  Lisäksi se
ei ole sidottu varsinaisesti ohjelmointikieliin, vaan sitä voidaan käyttää
moninaisiin asioihin (esimerkiksi kääntää Latex"-dokumentteja).  Sääntöjen
formaatti on:
\begin{verbatim}
kohde: lähdetiedostot
       komennot joilla lähteistä tehdään kohde
\end{verbatim}
Hitusen laajempi esimerkki Makefileistä löytyy sivulta:
\url{http://www.cs.colby.edu/maxwell/courses/tutorials/maketutor/}
Ja varsinainen täysi dokumentointi:
\url{https://www.gnu.org/software/make/manual/make.html}


Git on hajautettu versionhallintajärjestelmä.  Sillä voi helposti seurata
mitä muutoksia koodiin on tehty.  Hajautetulla tarkoitetaan että se toimii
myös jos useampi eri henkilö tekee muutoksia yhtäaikaa.  Jos muutokset
tulevat eri osiin koodia, ne yhdistyvät automaattisesti.  Hankalammat
tapaukset vaativat ihmisapua siivoamisessa.

Git"-projektin dokumentointi: \url{http://git-scm.com/documentation}


\section{Muiden lähdekoodit}

Useimmin joudutaan käyttämään jonkun muun kirjoittamaa lähdekoodia, joten
tutustutaan miten niitä käännetään.

\begin{enumerate}
\item Hae xbill"-ohjelman (\url{http://www.xbill.org/}) lähdekoodit:
\emph{wget http://www.xbill.org/download/xbill-2.1.tar.gz}
\item Pura paketti: \emph{tar xvzf xbill-2.1.tar.gz}
\item xbill käyttää autoconf"-mekanismia konfigurointiinsa, tämän tunnistaa
''configure''"-nimisen ohjelman läsnäolosta.  Suorita tämä ohjelma:
\emph{./configure}
\item Kokeile kääntää ohjelma komennolla: \emph{make}
\item Ikävä kyllä ohjelman lähdekoodissa on bugi, ja sieltä puuttuu vihje
linkata libXpm"-kirjasto mukaan.  Korjaa tämä avaamalla ''Makefile'' ja
lisäämällä LIBS-sanalla alkavalle riville listan loppuun ''-lXpm''.
\item Käännä ohjelma nyt (\emph{make}) ja suorita se: \emph{./xbill}

\subsection*{Sama uudestaan}

\item xbill tulee myös Debianin mukana.  Sen lähdekoodit voi ladata
komennolla: \emph{apt-get source xbill}
\item siirry xbill-2.1 -hakemistoon ja suorita: \emph{debuild}
\item Jos vilkaiset hakemistoon ''debian/patches'' löydät Debian"-projektin
alkuperäiseen jakeluun tekemät muutokset, kuten edellätekemämme
libXpm"-korjauksen.
\end{enumerate}

Prosessin pitäisi tehdä ylempään hakemistoon deb"-paketti jonka voit asentaa
(\emph{dpkg -i xbill\_2.1-8\_amd64.deb}), mutta samalla antaa varoitus että
sinulta puuttuu GPG"-allekirjoitusavain.  (GPG"-avaimet ovat osa Debianin
pakettien ehjyystarkistuksia, mitä ei tietenkään voida tehdä, kun emme ole
paketin virallisia ylläpitäjiä.) Debian"-paketointi on hieman isompi
kokonaisuus niellä, mutta jos haluatte kurkistaa jäniksenkoloon, on
Debian"-projektilla massiiviset määrät ohjeita osoitteessa:
\url{https://wiki.debian.org/Packaging}


\section{Git+CMake esimerkki}

Esimerkki vähän isomman projektin hakemisesta ja kääntämisestä.

\url{https://github.com/raceintospace/raceintospace}

\begin{enumerate}
\item Kloonaa projekti:
\emph{git clone https://github.com/raceintospace/raceintospace.git}
\item Asenna ohjelman tarvitsemat kirjastot:
\emph{sudo apt-get install cmake libsdl-dev libboost-dev libpng-dev
libjsoncpp-dev libogg-dev libvorbis-dev libtheora-dev libprotobuf-dev
protobuf-compiler}

\item Tehdään kansio jossa käännöstyö tehdään (ns. out"-of"-directory"-build):\\
\emph{mkdir raceintospace-build\\
cd raceintospace-build\\
cmake ../raceintospace\\
make}

Ohjelman pitäisi kääntyä ongelmitta.  Kääntäjä tulostaa useita varoituksia,
mutta ne eivät varsinaisesti estä toimintaa.

\item Jotta voisimme asentaa ohjelman /usr/local hakemistoon, lisää itsesi joko
''staff''"-ryhmään (muista että sinun täytyy logata ulos ja sisään jotta
ryhmäoikeusmuutokset tulevat voimaan).  Vaihtoehtoisesti voit suorittaa
seuraavat komennot \emph{sudo}-komennon avulla.
\item Kun ohjelma on käännetty, jos kiinnostaa kokeilla (se on vanha
avaruusohjelma"-peli), sen voi asentaa komennolla: \emph{make install}
\item Jos haluatte ohjelman mäkeen, se on asentunut kansioon
/usr/local/share/raceintospace ja /usr/local/bin/raceintospace:
\emph{rm -rf /usr/local/bin/raceintospace /usr/local/share/raceintospace}
\end{enumerate}


\end{document}
