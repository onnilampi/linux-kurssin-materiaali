\documentclass[12pt,portrait,a4]{article}
\usepackage[finnish]{babel}
\usepackage[utf8]{inputenc}
\usepackage{multicol}
\usepackage{calc}
\usepackage{ifthen}
\usepackage{geometry}
\usepackage{amsmath,amsthm,amsfonts,amssymb}
\usepackage{color,graphicx,overpic}
\usepackage{hyperref}
\usepackage{enumitem}


\pdfinfo{
  /Title (harkkaohje3.pdf)
  /Creator (TeX)
  /Producer (pdfTeX 1.40.0)
  /Author (Risto Järvinen ja CEC ry)
  /Subject (Harkkaohje3)
  /Keywords (pdflatex, latex,pdftex,tex)}

% This sets page margins to .5 inch if using letter paper, and to 1cm
% if using A4 paper. (This probably isn't strictly necessary.)
% If using another size paper, use default 1cm margins.
\ifthenelse{\lengthtest { \paperwidth = 11in}}
    { \geometry{top=.5in,left=.5in,right=.5in,bottom=.5in} }
    {\ifthenelse{ \lengthtest{ \paperwidth = 297mm}}
        {\geometry{top=1cm,left=1cm,right=1cm,bottom=1cm} }
        {\geometry{top=1cm,left=1cm,right=1cm,bottom=1cm} }
    }

% Turn off header and footer
\pagestyle{empty}

% Redefine section commands to use less space
\makeatletter
\renewcommand{\section}{\@startsection{section}{1}{0mm}%
                                {-1ex plus -.5ex minus -.2ex}%
                                {0.5ex plus .2ex}%x
                                {\normalfont\large\bfseries}}
\renewcommand{\subsection}{\@startsection{subsection}{2}{0mm}%
                                {-1explus -.5ex minus -.2ex}%
                                {0.5ex plus .2ex}%
                                {\normalfont\normalsize\bfseries}}
\renewcommand{\subsubsection}{\@startsection{subsubsection}{3}{0mm}%
                                {-1ex plus -.5ex minus -.2ex}%
                                {1ex plus .2ex}%
                                {\normalfont\small\bfseries}}
\makeatother

% Don't print section numbers
\setcounter{secnumdepth}{0}


\setlength{\parindent}{0pt}
\setlength{\parskip}{0pt plus 0.5ex}


% -----------------------------------------------------------------------

\begin{document}
\raggedbottom

\begin{multicols}{2}
\setlength{\premulticols}{1pt}
\setlength{\postmulticols}{1pt}
\setlength{\multicolsep}{1pt}
\setlength{\columnsep}{2pt}

Nimi:\hrulefill

%Op.num:\hrulefill

\end{multicols}

\section{Tekstieditori}

Kertausta viime kerrasta:
\begin{enumerate}
\item Luo kotihakemistoosi yksi tyhjä tiedosto, sekä uusi hakemisto.
\item Siirrä luomasi tiedosto luomaasi hakemistoon.
\item Avaa tyhjä tiedosto tekstieditorilla (kokeile nano ja vim, ohjeet
komentolistassa)
\item Tulosta muokkaamasi tiedosto terminaaliin.
\item Kokeile muuntaa tiedoston rivinvaihdot dos/unix-muotojen välillä.
Näetkö eroa tekstieditorissa?
\end{enumerate}

%Nanon käyttö?

%Merkistöesimerkki mallitiedostolla jossa on väärät rivinvaihdot?

\section{Ohjelmistojen asentaminen}

Debianissa ja muissa yleisissä jakeluissa on käytössä apt-pakettimanageri. Mikäli teet harkkoja järjestelmällä, josta aptitudea ei löydy (esim. Arch tai Gentoo), käytä vastaavaa pakettimanageria.
\begin{itemize}
\item Tarkista, mitä riippuvuuksia paketti ''htop'' tarvitsee. Listaa näistä muutama.
\item Asenna kyseinen paketti.
\item Käynnistä nettiselain.
\item Käynnistä htop, tapa äsken käynnistämäsi nettiselain.
\end{itemize}

\section{Palvelun asentaminen}

\begin{enumerate}
\item Asenna apache2 (''apt-get install apache2'')
\item Testaa toimiiko se (\url{http://localhost/})
\item Apachen käynnistäminen/sammuttaminen (vaativat sudon):
	\begin{itemize}
	\item apache2ctl (configtest / graceful)
	\item service apache2 start/stop
	\end{itemize}
\item Vaihda oletusportti 80=\textgreater 8080 (/etc/apache2/ports.conf).  Aja muutos sisään (lataa configuraatio uudelleen) ja testaa webbiselaimella \url{http://localhost:8080/}.
\item Kytke käyttöön käyttäjäkohtaiset hakemistot (\textasciitilde username/ = public\_html)
	\begin{itemize}
	\item Luo kotihakemistoosi hakemisto nimeltä ''public\_html''
	\item Kytke Apachen userdir-moduli päälle (sudo a2enmod userdir)
	\item Muuta Apachen oletusportti takaisin 8080=\textgreater 80 ja käynnistä Apache uudelleen.
	\item Testaa toimiiko (\url{http://localhost/~username/} )
	\item Kopioi ''/var/www/html/index.html'' ''public\_html'' hakemistoosi.
	\item Editoi itsellesi html-muotoinen ''index.html'' tiedosto
	\end{itemize}
%\item Tarkastele miten hakemisto-indeksit, käyttöoikeudet ja niiden
%asetusten periytyminen (.htaccess) toimii.

Huom: tällainen jako on saatavilla Aallon palvelimella. Kätevää, jos haluaa vaikka jakaa jotain koulun koneilla tehtyä julkisella linkillä eteen päin ilman bloatteja dropbox-kikkareita.
\url{http://users.aalto.fi/username/}


\item PHP-esimerkki
	\begin{itemize}
	\item Asenna php5 paketti.
	\item Kytke Apachen php5 moduuli päälle.
	\item Tee hello.php tiedosto ja cutpastea sinne sisältö sivulta:
\url{http://php.net/manual/en/tutorial.firstpage.php}
	\item Testaa php-skriptin toiminta webbiselaimella.
	\end{itemize}
\end{enumerate}

Jos et halua jättää Apachea ajoon, poista apache2-paketti.

\section{Järjestelmälogit}

\begin{enumerate}
\item Järjestelmän logit löytyvät hakemistopuusta: /var/log
\item Apachen logit /var/log/apache2/
\item Etsi/greppaa logeista:
	\begin{itemize}
	\item Sisäänloggautumiset
	\item Rootin loggautumiset
	\item Virheet
	\end{itemize}
\item Etsi aikaleiman mukaan jotain
\end{enumerate}

\section{Verkkoasetukset}

\begin{enumerate}
\item Taustoja IP-verkoista
	\begin{itemize}
	\item IP-osoitteet, subnetit ja laitareititys (gateway)
	\item arp / neighbor discovery
	\item Sisäiset osoitteet (10./8, 192.168./16, 172.162/12, 169.254./16)
	\end{itemize}
\item NetworkManageriin tutustuminen
	\begin{itemize}
	\item Tarkista IP-osoitteesi, mikä se on tällä hetkellä?
	\item Muuta IP-osoitteesi vapaavalintaiseksi uudeksi väliaikaiseksi osoitteeksi. (Luonnollisesti siten, että sama verkkomaski pätee, eikä verkko muutu.)
	\end{itemize}
\item Manuaalinen konfigurointi (networkmanagerin tappaminen)
	\begin{itemize}
	\item Debianissa verkkoasetukset: /etc/network
	\item Komentorivityökalu ip (legacy: ifconfig, arp, route)
	\item iwconfig (paketissa ''wireless-tools''), tämä komento ei toimi virtuaalikoneen kantta. Miksi ei? ;)
	\item Tutustu asetuksiisi ''ip''-komennolla, ja myös iwconfig jos
	käytössä on wlan-yhteys.
	\item IP-aliakset, lisää itsellesi osoite 192.168.10.10.
	\end{itemize}
%\item Verkkoyhteyden jakaminen?
\end{enumerate}


\end{document}
