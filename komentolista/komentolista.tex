\documentclass[10pt,portrait,a4]{article}
\usepackage[finnish]{babel}
\usepackage[utf8]{inputenc}
\usepackage{multicol}
\usepackage{calc}
\usepackage{ifthen}
%\usepackage[landscape]{geometry}
\usepackage{geometry}
\usepackage{amsmath,amsthm,amsfonts,amssymb}
\usepackage{color,graphicx,overpic}
\usepackage{hyperref}
\usepackage{enumitem}


\pdfinfo{
  /Title (komentolista.pdf)
  /Creator (TeX)
  /Producer (pdfTeX 1.40.0)
  /Author (Risto Järvinen ja CEC ry)
  /Subject (Komentolista)
  /Keywords (pdflatex, latex,pdftex,tex)}

% This sets page margins to .5 inch if using letter paper, and to 1cm
% if using A4 paper. (This probably isn't strictly necessary.)
% If using another size paper, use default 1cm margins.
\ifthenelse{\lengthtest { \paperwidth = 11in}}
    { \geometry{top=.5in,left=.5in,right=.5in,bottom=.5in} }
    {\ifthenelse{ \lengthtest{ \paperwidth = 297mm}}
        {\geometry{top=1cm,left=1cm,right=1cm,bottom=1cm} }
        {\geometry{top=1cm,left=1cm,right=1cm,bottom=1cm} }
    }

% Turn off header and footer
\pagestyle{empty}

% Redefine section commands to use less space
\makeatletter
\renewcommand{\section}{\@startsection{section}{1}{0mm}%
                                {-1ex plus -.5ex minus -.2ex}%
                                {0.5ex plus .2ex}%x
                                {\normalfont\large\bfseries}}
\renewcommand{\subsection}{\@startsection{subsection}{2}{0mm}%
                                {-1explus -.5ex minus -.2ex}%
                                {0.5ex plus .2ex}%
                                {\normalfont\normalsize\bfseries}}
\renewcommand{\subsubsection}{\@startsection{subsubsection}{3}{0mm}%
                                {-1ex plus -.5ex minus -.2ex}%
                                {1ex plus .2ex}%
                                {\normalfont\small\bfseries}}
\makeatother

% Define BibTeX command
\def\BibTeX{{\rm B\kern-.05em{\sc i\kern-.025em b}\kern-.08em
    T\kern-.1667em\lower.7ex\hbox{E}\kern-.125emX}}

% Don't print section numbers
%\setcounter{secnumdepth}{0}


\setlength{\parindent}{0pt}
\setlength{\parskip}{0pt plus 0.5ex}

%My Environments
\newtheorem{example}[section]{Example}
% -----------------------------------------------------------------------

\begin{document}
\raggedright
\footnotesize
\begin{multicols}{2}


% multicol parameters
% These lengths are set only within the two main columns
%\setlength{\columnseprule}{0.25pt}
\setlength{\premulticols}{1pt}
\setlength{\postmulticols}{1pt}
\setlength{\multicolsep}{1pt}
\setlength{\columnsep}{2pt}

\begin{center}
     \Large{\underline{Linux-komentoavuste}} \\
\tiny{Tekijät: Riba/Aalto ELEC/Comnet \& CEC ry}\\
\end{center}

\begin{description}[leftmargin=1.5cm,style=nextline]
\item[man] näyttää ohjelmien ohjekirjoja
\item[apropos] etsii hakusanaa ohjekirjoista
\end{description}

\section{Tiedostot ja hakemistot}

\begin{description}[leftmargin=1.6cm,style=nextline]
\item[ls] näyttää hakemiston sisällön
\item[cd] siirtyy hakemistoon
\item[pwd] näyttää tämänhetkisen hakemiston
\item[mkdir] luo hakemiston
\item[rmdir] poistaa hakemiston
\item[cp] kopioi tiedoston
\item[rm] poistaa tiedoston
\item[mv] siirtää tiedoston
\medskip
\item[df] kertoo paljonko levyillä on vapaata tilaa
\item[touch] päivittää tiedoston aikaleiman (tai luo tyhjän tiedoston)
\item[file] kertoo tiedoston tyypin
\item[ln] tekee kovan tai symbolisen linkin
\medskip
\item[dos2unix] Muuntaa dos-tyyliset rivinvaihdot unix-tyylisiksi
\item[unix2dos] Muuntaa unix-tyyliset rivinvaihdot dos-tyylisiksi
\end{description}

\section{Prosessien hallinta}

\begin{description}[leftmargin=1.9cm,style=nextline]
\item[ps] listaa prosesseja
\item[pstree] kauniimpi lista prosesseista
\item[kill] lähettää signaalin prosessille (yleensä pyyntö sammua)
\item[top] interaktiivinen prosessilista
\medskip
\item[uptime] kauanko järjestelmä on ollut päällä
\item[free] paljonko järjestelmässä on muistia
\medskip
\item[logout/exit] kirjautuu shellistä ulos
\item[reboot] käynnistää koneen uudelleen
\item[halt] sammuttaa koneen
\item[shutdown] sammuttaa/boottaa koneen
\end{description}

\section{Tiedon hakeminen}

Tässä yhteydessä tieto on joko teksti- tai binääritiedostoja.

\begin{description}[leftmargin=1.5cm,style=nextline]
\item[find] etsii tiedostoja
\item[locate] etsii tiedostoja (käyttää ennaltatehtyä tietokantaa)
\item[grep] etsii merkkijonoja syötteestä, esim tiedostoista
\item[xargs] muuttaa syötteen komentoparametreiksi toiselle komennolle
\medskip
\item[sort] järjestää syötetyt merkkijonot
\item[head] poimii tekstitiedostosta N ensimmäistä riviä
\item[tail] poimii tekstitiedostosta N viimeistä riviä
\medskip
\item[cat] tulostaa annetun lähteen
\item[less] lukee/selaa annettua lähdettä
\item[zless] lukee/selaa pakattua lähdettä
\item[dd] Siirtää tietoa annetusta lähteestä annettuun kohteeseen
\medskip
\item[echo] tulostaa annetun merkkijonon
\item[tee] jakaa tulosteen kahteen
\item[date] tulostaa aikaleiman
\item[wc] laskee montako merkkiä/sanaa/riviä syötteessä on
\end{description}

\section{Syötteen ja tulosteen ohjaaminen}

Jokainen komento voi ottaa yhden syötteen (stdin) ja antaa kaksi tulostetta;
tuloste (stdout) ja virhetuloste (stderr).  Oletuksena syöte on näppäimistö,
ja tulosteet tulostetaan konsolille.


\begin{description}[leftmargin=1.5cm,style=nextline]
\item[komento \textgreater tiedosto] ohjaa komennon tulosteen tiedostoon
\item[komento1 \textbar komento2] putkittaa komennon1 tulosteen komennon2 syötteeksi
\item[komento \textless tiedosto] ohjaa tiedoston komennon syötteeksi
\medskip
\item[komento 2\textgreater tiedosto] ohjaa komennon virhetulosteen tiedostoon.
\item[komento \textless tiedosto0 \textgreater tiedosto1 2\textgreater tiedosto2]
\medskip
\item[komento \textgreater /dev/null] ohjaa tulosteen ''roskiin''
\item[komento 2\textgreater\&1] Yhdistää virhetulosteen ja normitulosteen
\item[komento 2\textgreater\&1 \textgreater tiedosto] ohjaa virhetulosteen konsolille ja normitulosteen tiedostoon
\item[komento \textgreater tiedosto 2\textgreater\&1] ohjaa normitulosteen tiedostoon ja virhetulosteen sinne mukaan

\end{description}

\section{Käyttäjät ja oikeudet}

\begin{verbatim}
tyyppi
|omistajan oikeudet
||||ryhmän oikeudet
|||||||muiden oikeudet
-rwxrwxrwx
r = luku, w = kirjoitus, x = suorittaminen
\end{verbatim}

\begin{description}[leftmargin=1.5cm,style=nextline]
\item[chown] vaihtaa tiedoston omistajaa
\item[chgrp] vaihtaa tiedoston ryhmää
\item[chmod] muokkaa oikeuksia
\item[adduser] lisää käyttäjä järjestelmään
\item[deluser] poista käyttäjä järjestelmästä
\item[passwd] salasanan vaihtaminen
\medskip
\item[who/w] tulostaa keitä on kirjautuneina
\end{description}


\section{Tekstieditorit}

Pikainen ohjeistus vimin käytöstä:
\begin{itemize}
\item Tiedoston avaaminen: vim \textless polku tiedostoon\textgreater
\item Peruskäytössä vimissä on kaksi moodia, komentomoodi ja muokkausmoodi.
 Käynnistyessään vim on komentomoodissa, pääset muokkausmoodiin painamalla
i"-näppäintä (i=insert)
\item Palaaminen muokkaustilasta takaisin komentotilaan ESC-näppäimellä.
\item Tekstin hakeminen /
\item Kirjoita muutokset tiedostoon komennolla :w
\item Sulje editori komennolla :q
\item Kuten varmaan huomaat, komennot alkavat aina kaksoispisteellä.
\end{itemize}

Pikainen ohjeistus nanon käytöstä:
\begin{itemize}
\item Tiedoston avaaminen: nano \textless polku tiedostoon\textgreater
\item Ruudun alalaidassa on avusteet yleisimpiin komentoihin.
\item Tekstin hakeminen: CTRL-w
\item Kirjoita muutokset tiedostoon komennolla CTRL-o
\item Sulje editori komennolla CTRL-x
\end{itemize}

\section{Verkkoasetukset}

Tietokoneessa on verkkolaitteita, joilla on osoitteita.  Osoitteet
reititetään reititystaulun mukaan.

\begin{description}[leftmargin=1.8cm,style=nextline]
\item[ip link] Verkkolaitteet
\item[ip addr] Verkko-osoitteet
\item[ip neigh] Naapureiden tiedot
\item[ip route] Reititystiedot
\item[iwconfig] Langattoman verkon asetukset
\medskip
\item[ipcalc] laskee verkkopeitteitä ja osoitteita
\item[wget] Hakee tiedoston URL-osoitteesta
\medskip
\item[ping] testaa vastaako tietty noodi (nimellä tai IP-osoitteella)
\item[traceroute/mtr] tutkii reitin kohdenoodille
\item[tcpdump] kerää liikennettä verkkolaitteella liikkuu
\item[wireshark] graafinen tcpdump + protokolla-analysaattori
\item[nslookup/dig] tekee nimipalvelupyynnön
\end{description}

\section{Paketinhallinta}

\begin{description}[leftmargin=1.8cm,style=nextline]
\item[apt-get] Pakettienkäsittely-työkalu
\item[apt-cache] Tekee hakuja pakettitietokantoihin

\item[aptitude] Tekstipohjainen käyttöliittymä
\item[synaptics] Graafinen käyttöliittymä
\end{description}


\section{Pakkaustyökalut}

\begin{description}[leftmargin=1.5cm,style=nextline]
\item[tar] koostaa/purkaa tiedostot ja niiden oikeudet yhteen tiedostoon
\item[gzip] pakkaa tiedoston gz-muotoon
\item[gunzip] purkaa gz-tiedoston
\item[zip] koostaa ja pakkaa tiedostot zip-tiedostoon
\item[unzip] purkaa zip-tiedoston
\end{description}

\section{Screen pikaopas}

\begin{description}[leftmargin=1.5cm,style=nextline]
\item[ctrl-a c] luo uuden ikkunan
\item[ctrl-a n] siirtyy seuraavaan ikkunaan
\item[ctrl-a p] siirtyy edelliseen ikkunaan
\item[ctrl-a NUMERO] siirtyy ikkunaan NUMERO
\item[ctrl-a d] irroittautuu sessiosta
\item[exit] sulkee aktiivisen ikkunan/session
\item[screen -r] liittyy olemassaolevaan sessioon
\item[screen -rD] liittyy olemassaolevaan sessioon, katkaisten muut yhteydet
\item[screen -rx] liittyy olemassaolevaan sessioon, muiden rinnalle
\end{description}

ref \url{http://aperiodic.net/screen/quick_reference}

\section{SSH pikaopas}
\begin{description}[leftmargin=1.5cm,style=nextline]
\item[ssh host] aloittaa pääteyhteyden koneelle host
\item[ssh user@host] pääteyhteys käyttäjätunnuksella user
\item[ssh host komento] suorita komento koneella host
\item[ssh -i idfile host] pääteyhteys annetulla julkisella avaimella
\item[ssh-keygen] luo julkisen autetikointiavaimen
\item[ssh-copy-id -i idfile host] kopioi juokisen avaimen koneelle host
\item[sftp host] aloittaa tiedonsiirtoyhteyden
\item[scp tiedosto host] siirtää tiedoston koneelle host
\item[scp host:tiedosto] siirtää tiedoston koneelta host
\item[-L {[bindaddr:]}port:host:hostport] tekee tunnelin paikallisesta (osoitteella bindaddr) portista port kohteelle host:hostport
\item[-L {[bindaddr:]}port:host:hostport] tekee tunnelin etäkoneelta (osoitteella bindaddr) portista port paikalliselle kohteelle host:hostport
\end{description}

\section{Git pikaopas}
\begin{description}[leftmargin=1.8cm,style=nextline]
\item[git init] luo git-repositorion
\item[git clone url] kopioi git-repositorion osoitteesta \textit{url}
 \medskip
\item[git status] kertoo repositorion tilan
\item[git diff] näyttää repositoriossa tapahtuneet muutokset
\item[git add tiedosto] lisää tiedoston tulevaan kommittiin
\item[git commit -m ''lyhyt kuvaus''] tallentaa (kommitoi) muutoksen repositorioon
 \medskip
\item[git pull] lataa ja yhdistää lähderepositorioon tehdyt muutokset
\item[git pull --rebase] lataa lähderepositorion muutokset ja perustaa paikalliset muutokset sille
\item[git push] lähettää paikalliset tallennetut muutokset lähderepositorioon
 \medskip
\item[git branch] listaa paikallisen repositorion haarat
\item[git branch nimi] luo paikallisen haaran \textit{nimi}
\item[git checkout nimi] vaihtaa haaraan \textit{nimi} ja päivittää hakemiston sisällön
\item[git merge nimi] yhdistää haaran \textit{nimi} valittuun haaraan
\item[git branch -d nimi] poistaa haaran nimeltä \textit{nimi}
 \medskip
\item[git reset leima] poistaa \textit{leima} version jälkeen tehdyt muutokset, mutta säilyttää muokatut tiedostot
\item[git reset --hard leima] poistaa \textit{leima} version jälkeiset muutokset ja palauttaa hakemiston tilan taannoiseksi
 \medskip
\item[git blame tiedosto] tulostaa pyydetyn tiedoston merkiten kuka on viimeksi muokannut kutakin riviä
\end{description}



% You can even have references
%\rule{0.3\linewidth}{0.25pt}
%\scriptsize
%\bibliographystyle{abstract}
%\bibliography{refFile}
\end{multicols}
\end{document}
