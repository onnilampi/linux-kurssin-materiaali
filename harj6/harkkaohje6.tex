\documentclass[12pt,portrait,a4]{article}
\usepackage[finnish]{babel}
\usepackage[utf8]{inputenc}
\usepackage{multicol}
\usepackage{calc}
\usepackage{ifthen}
\usepackage{geometry}
\usepackage{amsmath,amsthm,amsfonts,amssymb}
\usepackage{color,graphicx,overpic}
\usepackage{hyperref}
\usepackage{enumitem}


\pdfinfo{
  /Title (harkkaohje6.pdf)
  /Creator (TeX)
  /Producer (pdfTeX 1.40.0)
  /Author (Risto Järvinen ja CEC ry)
  /Subject (Harkkaohje6)
  /Keywords (pdflatex, latex,pdftex,tex)}

% This sets page margins to .5 inch if using letter paper, and to 1cm
% if using A4 paper. (This probably isn't strictly necessary.)
% If using another size paper, use default 1cm margins.
\ifthenelse{\lengthtest { \paperwidth = 11in}}
    { \geometry{top=.5in,left=.5in,right=.5in,bottom=.5in} }
    {\ifthenelse{ \lengthtest{ \paperwidth = 297mm}}
        {\geometry{top=1cm,left=1cm,right=1cm,bottom=1cm} }
        {\geometry{top=1cm,left=1cm,right=1cm,bottom=1cm} }
    }

% Turn off header and footer
\pagestyle{empty}

% Redefine section commands to use less space
\makeatletter
\renewcommand{\section}{\@startsection{section}{1}{0mm}%
                                {-1ex plus -.5ex minus -.2ex}%
                                {0.5ex plus .2ex}%x
                                {\normalfont\large\bfseries}}
\renewcommand{\subsection}{\@startsection{subsection}{2}{0mm}%
                                {-1explus -.5ex minus -.2ex}%
                                {0.5ex plus .2ex}%
                                {\normalfont\normalsize\bfseries}}
\renewcommand{\subsubsection}{\@startsection{subsubsection}{3}{0mm}%
                                {-1ex plus -.5ex minus -.2ex}%
                                {1ex plus .2ex}%
                                {\normalfont\small\bfseries}}
\makeatother

% Don't print section numbers
\setcounter{secnumdepth}{0}

%\setlist{nosep}
\setlist{topsep=5pt}
\setlength{\parindent}{0pt}
\setlength{\parskip}{0pt plus 0.5ex}


% -----------------------------------------------------------------------

\begin{document}
\raggedbottom

\begin{multicols}{2}
\setlength{\premulticols}{1pt}
\setlength{\postmulticols}{1pt}
\setlength{\multicolsep}{1pt}
\setlength{\columnsep}{2pt}

Nimi:\hrulefill


\end{multicols}

\section{Ympäristömuuttujat}

Ympäristömuuttujat ovat komentotulkin (shell) levittämiä muuttujia, joilla
välitetään käyttäjän asetuksia.  Tässä harjoitellaan bash-tulkin syntaksia.

\begin{enumerate}
\item Tulosta ympäristömuuttujasi: \emph{printenv}
\item Tarkastele komentopolkuasi, joka löytyy muuttujasta \emph{PATH}:
\emph{printenv PATH} \\
Komentopolku on lista hakemistoja, joista komentoriviltä ajettavia komentoja
haetaan.
\item Tehdään esimerkiksi uusi muuttuja: \emph{export ESIM=helloworld}
\item Ympäristömuuttujia voidaan käyttää komentorivillä: \emph{echo \$ESIM}
tai \emph{echo \$\{ESIM\}}
\item Muuttuja voi määritellä komentokohtaisesti myös: \emph{VALIAIK=''heipa
hei'' echo \$VALIAIK}
\item Tee kotihakemistoosi hakemisto \emph{bin} ja lisää tämä
komentopolulle: \emph{export PATH=\$\{PATH\}:\textasciitilde/bin}
\end{enumerate}

\section{Komentotiedostot}

Komentotiedosto (shell script, batch file) on suoritettava tiedosto, jossa
on komentoja jotka komentotulkki suorittaa peräkkäin.  Komentotiedostot
tunnistetaan laittamalla tiedoston alkuun ns. hashbang-tunniste 
(\emph{\#!/bin/sh}), joka kertoo millä tulkilla komentotiedosto tulee
suorittaa.  Lisäksi komentotiedosto tulee laittaa suoritettavaksi
(\emph{chmod +x tiedosto}).  Jotkut nimeää komentotiedostot \emph{.sh}
päätteellä, mutta tämä ei ole pakollista.

Komentotiedostot ovat oma rajoittunut ohjelmointikielensä,
mutta niiden käyttökelpoisuus tehtävien automatisoinnissa on
kiistämätön.  Käsitellään niiden rakennetta tässä lyhyesti parilla
esimerkillä.

\begin{enumerate}
\item Luo bin-hakemistoosi tiedosto nimeltä \emph{mun\_ohjelmat.sh} ja syötä
sinne komennot:
\begin{verbatim}
#!/bin/sh
# Risuaidalla voidaan tehdä kommentteja komentotiedostoihin
# Tallennetaan muuttujaan toisen komennon ulostulo:
MINAITSE=`whoami`
# Toinen tapa tehdä sama asia:
# MINAITSE=$(whoami)
# Tulostetaan lista omista prosesseistani ja lasketaan rivien määrä
ps --no-headers -u $MINAITSE | wc -l
\end{verbatim}
Edellä whoami"-komennon kanssa käytetty merkki on ns. backtick, löytyy plus"- ja
backspace"-nappien välistä näppäilyllä: \emph{shift-` välilyönti}.  Näiden
kanssa tarkkana.
\item Lisää tiedostolle suoritusoikeus: \emph{chmod +x mun\_ohjelmat.sh}
\item Nyt voit suorittaa komennon \emph{mun\_ohjelmat.sh} ja se tulostaa
montako prosessia sinulla on ajossa.
\end{enumerate}

\section{Ohjausrakenteet}

Komentotiedostoissa on (komentotulkista riippuen; bash/tcsh/zsh) myös
ohjausrakenteita kuten if/then/else-puut ja for-silmukkoja.  Bashin
manuaalisivu on rasittavan laaja, mutta yksi hyvä täsmäluettava on lista
ehtolauseista luvussa ''CONDITIONAL EXPRESSIONS''.

\begin{enumerate}
\item Aloitetaan uuden komentotiedoston kirjoittaminen.  Lisää siihen
seuraavat komennot:
\begin{verbatim}
#!/bin/sh
# Tarkastetaan ettei komentotiedostoa saa ajaa root-käyttäjänä
if [ `whoami` = 'root' ]
then
    echo 'Do NOT run as root, dummy!'
    exit 1
else
    echo "Welcome, `whoami`."
    # Vihje: ero lainaus- (") ja heitto- (') merkkien välillä on että lainausmerkkien
    # välinen osa tulkitaan myös. Kokeile vaihtaa edeltävään komentoon heittomerkit.
fi
\end{verbatim}
\item Testaa ohjelmaa normaalikäyttäjänä ja root"-käyttäjänä (esim.
sudo"-komennolla).
\item Yleinen tilanne on että komentoja joudutaan ajamaan useita kertoja ja
tulokset halutaan eri tiedostoihin.  Tehdään näin esimerkinomaisesti
for"-silmukalla ja \emph{seq}"-komennolla.  Lisää edelliseen tiedostoon:
\begin{verbatim}
for i in `seq -w 1 10`
do
    date > timestamp${i}.txt
    echo Sleeping $i seconds.
    sleep $i
done
\end{verbatim}
\item Syntaksi on siis \emph{for MUUTTUJA in ARVOT.. ; do komentoja ; done}.
Jatketaan tiedostoa:
\begin{verbatim}
for each in timestamp*.txt ; do
    cat $each
    rm $each
done
echo All done.
\end{verbatim}

\end{enumerate}

\section{Komentoparametrit}

Komentotiedostoille voi antaa myös parametreja kun niitä kutsutaan
komentoriviltä.  Näihin viitataan \$NUMERO "-tyyliin.  Nolla on
suoritettavan komennon nimi, seuraavat siitä ylöspäin annetut parametrit. 
Esim:

\begin{verbatim}
#!/bin/sh
echo This script is `basename $0`
# Onko 
if [ "$1" = "" ] ; then
    echo Master, give me some parameters
    exit 0
fi
if [ -r "$1" ] ; then
    echo First parameter is $1 : `wc -l "$1"` lines
else
    echo There\'s no such file as $1
fi
\end{verbatim}



\end{document}
