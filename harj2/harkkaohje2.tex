\documentclass[12pt,portrait,a4]{article}
\usepackage[finnish]{babel}
\usepackage[utf8]{inputenc}
\usepackage{multicol}
\usepackage{calc}
\usepackage{ifthen}
\usepackage{geometry}
\usepackage{amsmath,amsthm,amsfonts,amssymb}
\usepackage{color,graphicx,overpic}
\usepackage{hyperref}
\usepackage{enumitem}


\pdfinfo{
  /Title (harkkaohje.pdf)
  /Creator (TeX)
  /Producer (pdfTeX 1.40.0)
  /Author (Risto Järvinen)
  /Subject (Harkkaohje)
  /Keywords (pdflatex, latex,pdftex,tex)}

% This sets page margins to .5 inch if using letter paper, and to 1cm
% if using A4 paper. (This probably isn't strictly necessary.)
% If using another size paper, use default 1cm margins.
\ifthenelse{\lengthtest { \paperwidth = 11in}}
    { \geometry{top=.5in,left=.5in,right=.5in,bottom=.5in} }
    {\ifthenelse{ \lengthtest{ \paperwidth = 297mm}}
        {\geometry{top=1cm,left=1cm,right=1cm,bottom=1cm} }
        {\geometry{top=1cm,left=1cm,right=1cm,bottom=1cm} }
    }

% Turn off header and footer
\pagestyle{empty}

% Redefine section commands to use less space
\makeatletter
\renewcommand{\section}{\@startsection{section}{1}{0mm}%
                                {-1ex plus -.5ex minus -.2ex}%
                                {0.5ex plus .2ex}%x
                                {\normalfont\large\bfseries}}
\renewcommand{\subsection}{\@startsection{subsection}{2}{0mm}%
                                {-1explus -.5ex minus -.2ex}%
                                {0.5ex plus .2ex}%
                                {\normalfont\normalsize\bfseries}}
\renewcommand{\subsubsection}{\@startsection{subsubsection}{3}{0mm}%
                                {-1ex plus -.5ex minus -.2ex}%
                                {1ex plus .2ex}%
                                {\normalfont\small\bfseries}}
\makeatother

% Don't print section numbers
\setcounter{secnumdepth}{0}


\setlength{\parindent}{0pt}
\setlength{\parskip}{0pt plus 0.5ex}


% -----------------------------------------------------------------------

\begin{document}
\raggedbottom

\begin{multicols}{2}
\setlength{\premulticols}{1pt}
\setlength{\postmulticols}{1pt}
\setlength{\multicolsep}{1pt}
\setlength{\columnsep}{2pt}

Nimi:\hrulefill

Op.num:\hrulefill

\end{multicols}

\section{Hakemistot}

\begin{enumerate}
\item Syötä komento ''\textbf{cd foobar}'', mitä tapahtui? \hrulefill

\item Syötä komento ''\textbf{cd ..}'' (huomaa välilyönti ennen pisteitä),
käytä ''\textbf{pwd}'' komentoa.  Mitä tapahtui? \\ . \hrulefill

\item Listaa hakemiston sisältö ''\textbf{ls}'' komennolla.  Mitä näet? 
\hrulefill

\item Syötä komento ''\textbf{cd}''.  Mitä tapahtui?  Käytä ''\textbf{pwd}''
komentoa.  \hrulefill

\item Syötä komento ''\textbf{cd ..}'' kahdesti.  Mikä on tämän hakemiston
sisältö ja rooli?  \\ .  \hrulefill

\item Syötä komento ''\textbf{cd root}'', mitä tapahtui?  \hrulefill

\item Syötä komento ''\textbf{ls -l}''.  Mihin hakemistoihin sinulla on pääsy?
 \hrulefill\\ . \hrulefill

\end{enumerate}

\section{Tiedostot}

\begin{enumerate}

\item Siirry hakemistoon ''\textbf{/}'' ja sitten hakemistoon ''\textbf{etc}''.  Syötä komento
''\textbf{ls}''.  Jos tuloste on pidempi kuin näyttösi, voit yrittää tehdä
ikkunastasi suurempaa tai käyttää shift-pageup ja shift-pagedown nappeja.

\item Syötä komento ''\textbf{file services}''.  Mikä on tämän tiedoston
tyyppi?  \hrulefill

\item Käytä komentoa ''\textbf{cat services}'' ja lue tiedosto.  Kokeile
komentoa ''\textbf{less services}''.  Mikä palvelu on portissa TCP/88? 
\hrulefill

\item Siirry takaisin kotihakemistoosi komennolla ''\textbf{cd}''

\item Syötä komento ''\textbf{file .}'', mikä on tämä tiedosto?  Voitko lukea sen sisältöä
''\textbf{cat .}'' komennolla? \\ . \hrulefill

\item Katso komennon ''\textbf{cp}'' ohjeet (''\textbf{man cp}'' ja/tai
''\textbf{cp -{}-help}'').  Kopioi tiedosto ''\textbf{/etc/services}''
kotihakemistoosi.

\item Katso ''\textbf{cat}'' komennon ohjeet.  Tulosta ''\textbf{services}'' tiedoston sisältö
lisäten siihen rivinumerot.  Millä parametreilla teit tämän? \hrulefill

\item Suorita ''\textbf{grep -v '\#' services}''.  Katso grep-komennon ohjekirjasta mitä
tämä tekee.  Lisää tähän tulosteeseen rivinumerot riveille jotka eivät ole
tyhjiä.  Syötä käyttämäsi komennot tähän: \\ . \hrulefill \\ . \hrulefill

\end{enumerate}

\section{Mistä löytyy lisää tietoa?}

\begin{enumerate}
\item Lue ''\textbf{man intro}''

\item Lue ''\textbf{man ls}'' - kaikilla komennoilla on ohjekirja

\item Syötä ''\textbf{apropos pwd}'' - suorittaa hakuja ohjekirjoihin

\item Lue ''\textbf{ls -{}-help}''

\item Syötä ''\textbf{help cd}'' - help antaa ohjeita shellin sisäisitä
komennoista.
\end{enumerate}

\section{Katsellaan ympäriinsä}

\begin{enumerate}
\item Siirry hakemistoon ''\textbf{/proc}''.  Tässä hakemistossa on käyttöjärjestelmän
ajonaikaista tilaa kuvaavia tiedostoja.
	\begin{itemize}
	\item Montako prosessoriydintä koneessasi on? \hrulefill
	\item Paljonko muistia koneessasi on? \hrulefill
	\item Paljonko swap-tilaa koneessasi on? \hrulefill
	\item Kauanko koneesi on ollut päällä? \hrulefill
	\end{itemize}

\item Siirry kansioon ''\textbf{/etc}''.  Tässä hakemistossa on käyttöjärjestelmän
asetustiedostoja.
	\begin{itemize}
	\item Montako käyttäjätunnusta koneessasi on?  Älä laske käsin, anna
koneen laskea. \hrulefill
	\item Montako käyttäjäryhmää vastaavasti? \hrulefill
	\item Missä tiedostossa koneen nimi säilytetään? \hrulefill
	\end{itemize}

\item Siirry hakemistoon ''\textbf{/usr/share/doc}''.  Tässä hakemistossa on
käyttöjärjestelmään asennettujen ohjelmien ohjekirjoja.  (Tämä on debianismi.)
	\begin{itemize}
	\item Nimeä 3 ohjelmaa jotka tulee paketin ''\textbf{coreutils}'' mukana (vihje: README.gz)
\\ . \hrulefill
	\item Mikä versio ''\textbf{bash}'' ohjelmasta koneessasi on?
\hrulefill
	\end{itemize}
\end{enumerate}

\section{Hakemistojen ja tiedostojen kanssa leikkimistä}

\begin{enumerate}
\item Luo uusi hakemisto kotihakemistoosi.
	\begin{itemize}
	\item Voitko siirtää tämän hakemiston samaan hakemistoon missä kotihakemistosi on?
\hrulefill
	\item Kopioi kaikki XPM-tiedostot ''\textbf{/usr/share/pixmaps}'' hakemistosta uuteen
hakemistoosi.  Mitä ''XPM'' tarkoittaa? \hrulefill
	\item Listaa XPM-tiedostot käänteisessä aakkosjärjestyksessä. Mikä
on kolmanneksi viimeisen tiedoston nimi? \hrulefill

	\end{itemize}

\item Siirry takaisin kotihakemistoosi.  Luo uusi hakemisto ja kopioi kaikki
tiedostot ''\textbf{/etc}'' hakemistosta luomaasi hakemistoon.  Kopioi myös
alihakemistot!  (Vihje: ''recursive'')
	\begin{itemize}
	\item Siirry uuteen hakemistoosi.  Tee uusi alihakemisto ja laita
sinne tiedostot, jotka alkavat isolla alkukirjaimella.  Tee toinen
alihakemisto ja laita sinne tiedostot, jotka alkavat pienellä kirjaimella. 
Mikä on pienin lukumäärä komentoja joilla tämän pystyt tekemään?  \hrulefill
	\item Poista ylijääneet tiedostot.
	\item Siirry kotihakemistoosi.  Poista kopioimasi tiedostot
sisältävä alihakemisto ja kaikki sen sisältämät tiedostot yhdellä
komennolla.  Tämä komento on: \hrulefill
	\end{itemize}

\item Tee kotihakemistoosi symbolinen linkki hakemistoon ''\textbf{/tmp}''.  Tarkasta
että se toimii.
	\begin{itemize}
	\item Tee kotihakemistoosi toinen linkki joka osoittaa edelliseen kotihakemistoosi
tekemään linkkiin.  Toimiiko se? \hrulefill
	\item Poista ensimmäinen linkki ja tarkasta kotihakemistosi sisältö.
Mitä tapahtui toiselle linkille? \\ . \hrulefill
	\end{itemize}
\end{enumerate}


\end{document}
