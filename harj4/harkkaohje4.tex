\documentclass[12pt,portrait,a4]{article}
\usepackage[finnish]{babel}
\usepackage[utf8]{inputenc}
\usepackage{multicol}
\usepackage{calc}
\usepackage{ifthen}
\usepackage{geometry}
\usepackage{amsmath,amsthm,amsfonts,amssymb}
\usepackage{color,graphicx,overpic}
\usepackage{hyperref}
\usepackage{enumitem}


\pdfinfo{
  /Title (harkkaohje4.pdf)
  /Creator (TeX)
  /Producer (pdfTeX 1.40.0)
  /Author (Risto Järvinen)
  /Subject (Harkkaohje4)
  /Keywords (pdflatex, latex,pdftex,tex)}

% This sets page margins to .5 inch if using letter paper, and to 1cm
% if using A4 paper. (This probably isn't strictly necessary.)
% If using another size paper, use default 1cm margins.
\ifthenelse{\lengthtest { \paperwidth = 11in}}
    { \geometry{top=.5in,left=.5in,right=.5in,bottom=.5in} }
    {\ifthenelse{ \lengthtest{ \paperwidth = 297mm}}
        {\geometry{top=1cm,left=1cm,right=1cm,bottom=1cm} }
        {\geometry{top=1cm,left=1cm,right=1cm,bottom=1cm} }
    }

% Turn off header and footer
\pagestyle{empty}

% Redefine section commands to use less space
\makeatletter
\renewcommand{\section}{\@startsection{section}{1}{0mm}%
                                {-1ex plus -.5ex minus -.2ex}%
                                {0.5ex plus .2ex}%x
                                {\normalfont\large\bfseries}}
\renewcommand{\subsection}{\@startsection{subsection}{2}{0mm}%
                                {-1explus -.5ex minus -.2ex}%
                                {0.5ex plus .2ex}%
                                {\normalfont\normalsize\bfseries}}
\renewcommand{\subsubsection}{\@startsection{subsubsection}{3}{0mm}%
                                {-1ex plus -.5ex minus -.2ex}%
                                {1ex plus .2ex}%
                                {\normalfont\small\bfseries}}
\makeatother

% Don't print section numbers
\setcounter{secnumdepth}{0}


\setlength{\parindent}{0pt}
\setlength{\parskip}{0pt plus 0.5ex}


% -----------------------------------------------------------------------

\begin{document}
\raggedbottom

\begin{multicols}{2}
\setlength{\premulticols}{1pt}
\setlength{\postmulticols}{1pt}
\setlength{\multicolsep}{1pt}
\setlength{\columnsep}{2pt}

Nimi:\hrulefill

Op.num:\hrulefill

\end{multicols}

\section{Screen}

Screen mahdollistaa terminaalisessioiden pitämisen elossa
vaikka terminaaliyhteys olisi poikki.

\begin{enumerate}
\item Luo uusi sessio nimelle ''hakku'': \emph{screen -S hakku}
\item Käynnistä tekstieditori luodussa sessiossa ja luo tiedosto hakku.txt
\item Katkaise yhteys sessioon: \emph{CTRL-a d}
\item Luo toinen sessio nimelle ''kuokka'' ja käynnistä sen sisään aptitude: \emph{screen -S kuokka aptitude}
\item Listaa mitä screen-sessioita järjestelmässä on: \emph{screen -ls}
\item Katkaise yhteys ''kuokkaan'' ja liitä yhteys ''hakkuun''
\item Luo ''hakkuun'' toinen ikkuna: \emph{CTRL-a c}
\item Liitä ''kuokka'' ''hakun'' toiseen ikkunaan: \emph{screen -r kuokka}
\item Sulje ''kuokasta'' aptitude.  Tällöin se screen sulkeutuu, koska sen
viimeinen ikkuna poistui.
\end{enumerate}

HUOM: älä avaa screenejä itsensä sisällä.  Silmukat räjähtävät.

\section{SSH}

SSH on kryptografisesti suojattu terminaali-etäkäyttö-ohjelma.

\begin{enumerate}
\item Muodosta SSH-yhteys palvelimelle.  Esimerkki: \emph{ssh username@kosh.aalto.fi}
\item Nyt voit suorittaa terminaalikomentoja palvelimella paikallisen koneen sijaan.
\item SSH+Screen on äärimmäisen joustava työkalu yhdistämän
terminaali-työskentely-sessiot koneelta toiselle siirtyessä.
\end{enumerate}

SSH-asiakasohjelmia on useimmille alustoille:
\begin{itemize}
\item putty windowsilla
\item openssh linux/unix/osx:llä
\item JuiceSSH androidilla
\end{itemize}

SSH on pääasiallinen työkalu verkkopalvelimien säätämiseen.

\subsection{Tiedostojen siirtäminen}

SSH:n kylkiäisapuohjelmat sisältää tiedostojen siirtämisen:
\begin{itemize}
\item puttyllä pscp/psftp, linuxilla scp/sftp.  GUI-softia filezilla, winscp.
\item Siirrä tekstitiedosto linux-koneeltasi jollekin toiselle koneelle.
\item Jos Linux-koneesi pyörii virtuaalikoneessa, siirrä tekstitiedosto
isäntäkoneelle.  Miltä tiedoston sisältö näyttää kun sen avaa isäntäkoneella?
(esim notepad windowsilla)
\end{itemize}

\subsection{Tunnelointi}

SSH voi ''tunneloida'' graafisia sovelluksia.  Tämä vaatii X11-palvelimen
koneeltasi.  Linuxissa X11 on mukana, MacOSX tarvii sen asentamisen
lisäpakettina, Windows tarvitsee erillissoftan.

\begin{enumerate}
\item Muodosta ssh-yhteys käyttäen ''-X'' parametria.  Esim. \emph{ssh -X username@brute.aalto.fi}
\item Käynnistä graafinen sovellus: \emph{xclock}
\item Ohjelma suoritetaan tällöin koneella johon otit yhteyden, mutta sen
graafinen käyttöliittymä piirtyy omalle koneellesi SSH-tunnelin toiseen päähän.
\item Tämä toimii myös isommillekin sovelluksille: \emph{matlab}
\item Siirtoyhteydestä aiheutuu jonkun verran yleisrasitetta toimintaan, että
monimutkaisemmat/graafisemmat sovellukset eivät toimi erityisen hyvin.
\end{enumerate}

SSH voi tunneloida myös geneerisesti TCP-yhteyksiä.
TCP-tunneleita on kahta tyyppiä: paikalliset ja etäpään tunnelit.

\begin{itemize}
\item Paikalliset tunnelit tarkoittaa että SSH kuuntelee paikallista TCP-porttia
ja välittää yhteyden etäpään kautta nimettyyn kohteeseen.
\item Etäpääntunnelit toimivat päinvastoin; SSH kuuntelee etäpäässä TCP-porttia
ja välittää yhteyden paikallisen koneen kautta nimettyyn kohteeseen.
\end{itemize}

Yksi käytännöllinen sovellus SSH-tunneleihin on kiertää palomuurirajoituksia.
Esimerkki tällaisesta olisi käyttää jostain Aalto-yliopiston verkon
ulkopuolelta Aallon wwwproxy.hut.fi palvelua Koshin kautta.

\begin{enumerate}
\item Tällöin tunneli muodostettaisiin seuraavasti:
\emph{ssh username@kosh.aalto.fi -L localhost:8080:wwwproxy.hut.fi:80}
\item Aseta tämän jälkeen webbiselaimesi proxy-asetukseksi: \emph{localhost:8080}
(firefox: preferences/advanced/network/settings)
\item Vahvista toiminta vaikkapa \url{http://whatismyipaddress.com}
\end{enumerate}

\subsection{Julkisen avaimen autentikointi}

SSH-yhteydet yksinkertaisimmillaan ovat salasanalla autentikoituja.
Salasanan tietäminen vahvistaa että käyttäjä on kyseisen käyttäjätunnuksen
omistaja.  Julkisen avaimen autentikointimenetelmässä muodostetaan ns.
julkinen avain ja ns. yksityinen avain.  Yksityisen avaimen omistaminen
vahvistaa käyttäjän identiteetin.

\begin{enumerate}
\item Suorita komento: \emph{ssh-keygen -f testi}
\item Tämä luo ''testi''-nimisen yksityisen avaimen, ja sitä vastaavan julkisen
avaimen ''testi.pub''.  Julkinen avain voidaan vapaasti kopioida palvelimelle
(sen tietäminen ei auta mahdollisia hakkereita varastamaan tunnuksiasi).
\item Tämän voi tehdä joko manuaalisesti kopioimalla ''testi.pub'' palvelimella
kotihakemistoosi ''.ssh/authorized\_hosts'' tiedostoon.  Tai automaattisesti
komennolla: \emph{ssh-copy-id -i testi username@palvelin}
\item Nyt voit kirjautua ilman salasanaa: \emph{ssh -i testi username@palvelin}
\end{enumerate}

Jos et käytä ''-f'' parametria niin julkinen avain luodaan profiiliisi
oletusavaimeksi ja sitä käytetään automaattisesti (ilmankin ''-i''
parametria).

Julkisen avaimen menetelmiä käytetään usein myös mahdollistamaan
automaattiset ylläpitotoimet ssh-yhteyksiä käyttäen; palvelinohjelmisto
käyttää ssh-avainta ottaakseen yhteydet.  Ei tarvita ihmistä tekemään
jokaikistä asiaa.

\section{Verkkodiagnostiikat}

Ethernet/WirelessLAN-verkoissa toimintamalli on että ''naapurit'', eli samassa
segmentissä ja subnetissä olijat, ovat suoraan saavutettavissa.  Pääsy
muuhun verkkoon tapahtuu ''gateway'' noodin läpi.  Gateway on reititin, joka
viestittelee muiden reitittimien kanssa kunnes viesti saadaan perille.

\subsection{Ping-komento}
\begin{enumerate}
\item Selvitä \emph{ip addr} komennolla oma osoitteesi.  Vastaako tämä pingiin?
\item Selvitä \emph{ip route} komennolla gatewaysi.  Vastaako tämä pingiin?
\item Selvitä ''www.aalto.fi'' palvelimen osoite nslookup -komennolla.
\item Testaa vastaako ''www.aalto.fi'' ping komentoon.  Mikä on keskimääräinen
vasteaika (rtt avg)?  Onko tuloksesi sama kuin muilla?
\end{enumerate}

Pää pyörällä IP-osoitteiden ja subnettien kanssa?  Numerojen pyörittämiseen
on näppärä apuohjelma nimeltä \emph{ipcalc}.  Asenna tämä (\emph{apt-get
install ipcalc}).  Poimi \emph{ip addr} komennon tulosteesta IPv4 osoitteesi
(inet -sanan jälkeen ''x.x.x.x/x'') ja syötä tämä tyyliin \emph{ipcalc x.x.x.x/x}.

\subsection{Reitinselvitys}
\begin{enumerate}
\item Millaisen näkymän antaa \emph{traceroute www.google.fi}?
Entä \emph{traceroute www.aalto.fi}?
\item Kokeile \emph{mtr} komentoa samoille kohteille.
\end{enumerate}


\subsection{Liikenteen dumppaaminen}

Seuraavat vaativat ylimääräisiä oikeuksia, joten suorita komennot joko
roottina tai sudo-komennon avulla.
\begin{enumerate}
\item Samalla kun suoritat yhdessä ikkunassa ping-komentoa, aja toisessa ikkunassa
\emph{sudo tcpdump -i verkkolaite icmp} (missä verkkolaite on käyttämäsi laite)
\item Suorita \emph{sudo tcpdump -i verkkolaite tcp port 80} ja lataa joku http:tä
käyttävä webbisivu, esimerkiksi \url{http://www.aalto.fi/} (https käyttää eri porttia).
\end{enumerate}

Wireshark on graafinen työkalu, joka voi dumpata liikennetta ja tarkastella
sen sisältöä.  Näiden työkalujen käytöstä lisää \emph{ELEC-C7240 Internet-tekniikat}.



\end{document}
